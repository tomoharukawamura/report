\documentclass[pdflatex,ja=standard,fleqn]{bxjsarticle}
\usepackage{ascmac,amsmath,amssymb,type1cm, tikz, graphicx}
\usetikzlibrary{intersections,calc,arrows.meta}

\title{熱工学課題2}
\author{J4-210447 川村朋広}
\begin{document}
\maketitle

\section*{1}
求める熱量を$\dot{q}$と置くと
\begin{eqnarray}
    h^{\prime\prime}\dot{M}&=&26755.7[kJ/s]\\&=&27[MW]
\end{eqnarray}
と求められる。
\section*{2}
単位時間当たりの熱量変化は
\begin{eqnarray*}
    -\dot{Q}=-10[MW]
\end{eqnarray*}
である。一方、比エンタルピーの値を$h$とおくと、
\begin{eqnarray}
    h&=&(1-x)\dot(h^{\prime})+xh^{\prime\prime}\\
    &=&(1-x)(h^{\prime\prime}-r)+xh^{\prime\prime}\\
    &=&h^{\prime\prime}-(h^{\prime\prime}-h^{\prime})(1-x)
\end{eqnarray}
より、比エンタルピーの変化量$dh$は、
\begin{eqnarray*}
    dh=-\dot{M}(h^{\prime\prime}-h^{\prime})(1-x)
\end{eqnarray*}
である。圧力一定なので、
\begin{align}
    &\dot{Q} = dh\\
    &\dot{M}(h^{\prime\prime}-h^{\prime})(1-x)=\dot{Q}\\
    &x=1-\frac{\dot{Q}}{\dot{M}h^{\prime\prime}-h^{\prime}}\\
    &x=0.56
\end{align}
\section*{3}
求めるエントロピーを$s$と置くと、
\begin{eqnarray}
    s&=&(1-x)s^{\prime}+xs^{\prime\prime}\\
    &=&s^{\prime}+\frac{x(h^{\prime\prime}-h^{\prime})}{T}\\
    &=&4.7
\end{eqnarray}
と求められる。
\end{document}