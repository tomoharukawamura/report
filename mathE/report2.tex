\documentclass[pdflatex,ja=standard,fleqn]{bxjsarticle}
\usepackage{ascmac,amsmath,amssymb,type1cm,graphicx}
\title{数学IEレポート2}
\author{J4-210447 川村朋広}
\begin{document}
\maketitle
\section*{問1}
\subsection*{(a)}
\begin{screen}
    2次元平面上の放物線$\boldsymbol{r}=(t,t^2)$について、原点から$t=a$までの距離を求めよ。
\end{screen}
【解答】\\
求める長さを$L(a)$とおくと、
\begin{eqnarray*}
    L(a)&=&\int_{0}^{a} \frac{d\boldsymbol{r}}{dt}dt\\
    &=&\int_{0}^{a} \sqrt{1+\frac{dt^2}{dt}}dt\\
    &=&\int_{0}^{a} \sqrt{1+2t}dt\\
    &=&\left[\frac{1}{3}(1+2t)^{\frac{3}{2}}\right]^a_0\\
    &=&\frac{1}{3} \left\{(1+2a)^{\frac{3}{2}}-1 \right\}
\end{eqnarray*}
と求められる。
\subsection*{(b)}
\begin{screen}
    $\boldsymbol{p}$は定ベクトル、$\boldsymbol{r}=(x,y,z)$、$r=|\boldsymbol{r}|$とし、3次元空間上のスカラー場
\begin{eqnarray*}
    \phi(\boldsymbol{r})=\frac{\boldsymbol{p}\cdot\boldsymbol{r}}{r^3}
\end{eqnarray*}
の勾配が、次式で与えられることを示せ。
\begin{eqnarray*}
    \mathrm{grad}\phi=\frac{r^2\boldsymbol{p}-3(\boldsymbol{p}\cdot\boldsymbol{r})\boldsymbol{r}}{r^5}
\end{eqnarray*}
\end{screen}
【証明】\\
$\boldsymbol{p}$の各成分を$(p_{x},p_{y},p_{z})$とする。
$\phi(r)$を$x$で微分した時を考える。
\begin{eqnarray*}
    \frac{\partial\phi}{\partial x}&=&\frac{\partial}{\partial x}\frac{\boldsymbol{p}\cdot\boldsymbol{r}}{(x^2+y^2+z^2)^{\frac{3}{2}}}\\
    &=&\frac{(x^2+y^2+z^2)^\frac{3}{2}\boldsymbol{p}\cdot\frac{\partial \boldsymbol{r}}{\partial x}-\frac{3}{2}(x^2+y^2+z^2)^{\frac{1}{2}}2x(\boldsymbol{p}\cdot\boldsymbol{r})}{(x^2+y^2+z^2)^3}\\
    &=&\frac{r^3p_{x}-3r(\boldsymbol{p}\cdot\boldsymbol{r})x}{r^6}\\
    &=&\frac{r^2p_{x}-3(\boldsymbol{p}\cdot\boldsymbol{r})x}{r^5}
\end{eqnarray*}
となる。$y$、$z$についても同様に微分すると
\begin{align*}
    &\frac{\partial\phi}{\partial y}=\frac{r^2p_{y}-3(\boldsymbol{p}\cdot\boldsymbol{r})y}{r^5}\\
    &\frac{\partial\phi}{\partial z}=\frac{r^2p_{z}-3(\boldsymbol{p}\cdot\boldsymbol{r})z}{r^5}
\end{align*}
が得られるので、題意の数式は示された。
\subsection*{(c)}
\begin{screen}
    3次元空間上の2つのベクトル場$\boldsymbol{a}(\boldsymbol{r})$、$\boldsymbol{b}(\boldsymbol{r})$について以下の恒等式を示せ。
    \begin{eqnarray*}
        \mathrm{rot}(\boldsymbol{a}\times\boldsymbol{b})=-(\mathrm{div}\boldsymbol{a})\boldsymbol{b}+(\mathrm{div}\boldsymbol{b})\boldsymbol{a}-(\boldsymbol{a}\cdot\nabla)\boldsymbol{b}+(\boldsymbol{b}\cdot\nabla)\boldsymbol{a}
    \end{eqnarray*}
\end{screen}
【証明】\\
$\boldsymbol{a}$の各成分を$(a_{x},a_{y},a_{z})$、$\boldsymbol{b}$の各成分を$(b_{x},b_{y},b_{z})$とおく。
\begin{eqnarray*}
    \boldsymbol{a}\times\boldsymbol{b}=(a_{y}b_{z}-a_{z}b_{y},a_{z}b_{x}-a_{x}b_{z},a_{x}b_{y}-a_{y}b_{x})
\end{eqnarray*}
なので、$x$成分について
\begin{equation*}
    \begin{split}
        &\left[\nabla\times(\boldsymbol{a}\times\boldsymbol{b})\right]_x\\
        &=\frac{\partial}{\partial y}(a_{x}b_{y}-a_{y}b_{x})-\frac{\partial}{\partial z}(a_{z}b_{x}-a_{x}b_{z})\\
        &=b_{y}\frac{\partial a_{x}}{\partial y}+a_{x}\frac{\partial b_{y}}{\partial y}-b_{x}\frac{\partial a_{y}}{\partial y}-a_{y}\frac{\partial b_{x}}{\partial y}-b_{x}\frac{\partial a_{z}}{\partial z}-a_{z}\frac{\partial b_{x}}{\partial z}+b_{z}\frac{\partial a_{x}}{\partial z}+a_{x}\frac{\partial b_{z}}{\partial z}\\
        &=-\left(a_{y}\frac{\partial}{\partial y}+a_{z}\frac{\partial}{\partial z}\right)b_{x}+\left(a_{y}\frac{\partial}{\partial y}+a_{z}\frac{\partial}{\partial z}\right)a_{x}-\left(\frac{\partial a_{y}}{\partial y}+\frac{\partial a_{z}}{\partial z}\right)b_{x}+\left(\frac{\partial b_{y}}{\partial y}+\frac{\partial b_{z}}{\partial z}\right)a_{x}\\
        &=-\left(a_{x}\frac{\partial}{\partial x}+a_{y}\frac{\partial}{\partial y}+a_{z}\frac{\partial}{\partial z}\right)b_{x}+\left(b_{x}\frac{\partial}{\partial x}+a_{y}\frac{\partial}{\partial y}+a_{z}\frac{\partial}{\partial z}\right)a_{x}\\
        &\quad-\left(\frac{\partial a_{x}}{\partial x}+\frac{\partial a_{y}}{\partial y}+\frac{\partial a_{z}}{\partial z}\right)b_{x}+\left(\frac{\partial b_{x}}{\partial x}+\frac{\partial b_{y}}{\partial y}+\frac{\partial b_{z}}{\partial z}\right)a_{x}\\
        &=\left[-(\mathrm{div}\boldsymbol{a})\boldsymbol{b}+(\mathrm{div}\boldsymbol{b})\boldsymbol{a}-(\boldsymbol{a}\cdot\nabla)\boldsymbol{b}+(\boldsymbol{b}\cdot\nabla)\boldsymbol{a}\right]_x
    \end{split}
\end{equation*}
$y$、$z$成分おいても同様に示せる。
\subsection*{(d)}
\begin{screen}
    ベクトル場$\boldsymbol{a}(\boldsymbol{r})=(y^2z^3,2xyz^3,3xy^2z^2)$について、曲線$C$:$\boldsymbol{r}=(t^2,t^4,t^6)$に沿って$t=0$から$t=1$まで変化させた際の線積分
    \begin{eqnarray*}
        \int_C \boldsymbol{a}\cdot d\boldsymbol{r}
    \end{eqnarray*}
    を求めよ
\end{screen}
【解答】\\
ベクトル場$\boldsymbol{a}(\boldsymbol{r})$をパラメータ$t$を用いて表すと
\begin{eqnarray*}
    \boldsymbol{a}(t)=(t^{26},2t^{24},3t^{22})
\end{eqnarray*}
となる。したがって、
\begin{eqnarray*}
    \int_C \boldsymbol{a}\cdot d\boldsymbol{r}&=&\int_{0}^{1} \boldsymbol{a}(t)\cdot\frac{d\boldsymbol{r}}{dt}dt\\
    &=&\int_{0}^{1} (2+8+18)t^{27}dt\\
    &=&\int_{0}^{1} 28t^{27}dt\\
    &=&\left[t^{28}\right]_{0}^{1}\\
    &=&1
\end{eqnarray*}
と求まる。
\subsection*{(e)}
\begin{screen}
    2次元平面のベクトル場$\boldsymbol{a}=(-\frac{y}{r^2},\frac{x}{r^2})\left[r=|\boldsymbol{r}|,\boldsymbol{r}=(x,y)\right]$について原点を中心とする半径$R$の円$C$に沿って一周した線積分
    \begin{eqnarray*}
        \oint_C \boldsymbol{a}\cdot d\boldsymbol{r}
    \end{eqnarray*}
    を、(1)直接求めるのと(2)グリーンの定理を用いるのでは結果が異なる。これを確かめ理由を考察せよ。
\end{screen}
【解答】
\begin{align*}
    &x=R{\rm cos}\theta\\
    &y=R{\rm sin}\theta
\end{align*}
とおく。\\
(1)直接求める場合\par
この時、$r=R$とする
\begin{eqnarray*}
    \oint_C \boldsymbol{a}\cdot d\boldsymbol{r}&=&\int_{0}^{2\pi} \boldsymbol{a}\cdot\frac{d\boldsymbol{r}}{d\theta}d\theta\\
    &=&\int_{0}^{2\pi} \begin{pmatrix}-\frac{{\rm sin}\theta}{R}\\\frac{{\rm cos}\theta}{R}\end{pmatrix}\cdot\begin{pmatrix}-R{\rm sin}\theta\\R{\rm cos}\theta\end{pmatrix}d\theta\\
    &=&\int_{0}^{2\pi} ({\rm sin^2}\theta+{\rm cos^2}\theta)d\theta\\
    &=&\int_{0}^{2\pi} d\theta\\
    &=&2\pi
\end{eqnarray*}
(2)グリーンの定理を用いる場合\par
円$C$の内部の領域を$D$とおく。
\begin{eqnarray*}
    \oint_C \boldsymbol{a}\cdot d\boldsymbol{r}&=&\oint_C \left(-\frac{y}{r^2}dx+\frac{x}{r^2}dy\right)\\
    &=&\iint_D \left(\frac{\partial}{\partial x}\frac{x}{r^2}+\frac{\partial}{\partial y}\frac{y}{r^2}\right)dxdy\\
    &=&\iint_D \left(\frac{r^2-2x^2}{r^4}+\frac{r^2-2y^2}{r^4}\right)dxdy\\
    &=&\iint_D \left(\frac{2r^2-2(x^2+y^2)}{r^4}\right)dxdy\\
    &=&\iint_D \left(\frac{2r^2-2r^2}{r^4}\right)dxdy\\
    &=&0
\end{eqnarray*}
上記の2つの手法で結果が異なるのは、原点$(0,0)\in D$において、被積分関数が連続でないからだといえる。
\subsection*{(f)}
\begin{screen}
    原点を中心とした、底面の半径$1$、高さ$2$の円柱に囲まれた領域を$T$とする。$T$の境界$S$に沿ったベクトル場$\boldsymbol{a}(\boldsymbol{r})=(xy^2,x^2y,y^2)$の面積分
    \begin{eqnarray*}
        \iint_S \boldsymbol{a}(\boldsymbol{r})\cdot d\boldsymbol{S}
    \end{eqnarray*}
    を考える。面積要素ベクトル$d\boldsymbol{S}$は外向きにとる。\par
    (f-1)底面($x^2+y^2\leq1,z=-1$)での面積分をもとめよ。\\
    (f-2)側面($x^2+y^2=1,-1\leq z\leq1$)での面積分を求めよ。\\
    (f-3)3重積分
    \begin{eqnarray*}
        \iiint_T (\mathrm{div}\boldsymbol{a})dv
    \end{eqnarray*}
    \quadを求め、ガウスの定理が成り立つことを確認せよ。
\end{screen}
【解答】\\
(f-1)\\
ベクトル$\boldsymbol{r}_{1}=(x,y,z)=(r{\rm cos}\theta,r{\rm sin}\theta,-1)$とおく。このとき
\begin{align*}
    &\frac{\partial \boldsymbol{r}_{1}}{\partial r}=\begin{pmatrix}{\rm cos}\theta\\{\rm sin}\theta\\0\end{pmatrix}\\
    &\frac{\partial \boldsymbol{r}_{1}}{\partial \theta}=\begin{pmatrix}-r{\rm sin}\theta\\r{\rm cos}\theta\\0\end{pmatrix}
\end{align*}
したがって、面積要素ベクトル$d\boldsymbol{S}$は
\begin{eqnarray*}
    d\boldsymbol{S}&=&\left(\frac{\partial \boldsymbol{r}_{1}}{\partial r}\times\frac{\partial \boldsymbol{r}_{1}}{\partial \theta}\right)drd\theta\\
    &=&\begin{pmatrix}0\\0\\r\end{pmatrix}drd\theta
\end{eqnarray*}
と求められる。向きを考慮すると、
\begin{eqnarray*}
    d\boldsymbol{S}=\begin{pmatrix}0\\0\\-r\end{pmatrix}drd\theta
\end{eqnarray*}
である。
\begin{eqnarray*}
    \iint_{x^2+y^2\leq 1,z=-1} \boldsymbol{a}(\boldsymbol{r})\cdot d\boldsymbol{S}&=&\iint_{0\leq r\leq1,0\leq\theta\leq 2\pi} \begin{pmatrix}r^3({\rm cos}\theta{\rm sin^2}\theta)\\r^3({\rm cos^2}\theta{\rm sin}\theta)\\r^2{\rm sin^2}\theta\end{pmatrix}\cdot\begin{pmatrix}0\\0\\-r\end{pmatrix}drd\theta\\
    &=&\int_{0}^{2\pi} {\rm sin^2}\theta d\theta \int_{0}^{1} -r^3dr\\
    &=&\int_{0}^{2\pi} \frac{1-{\rm cos}2\theta}{2} d\theta\int_{0}^{1} -r^3dr\\
    &=&-\frac{\pi}{4}
\end{eqnarray*}
(f-2)\\
ベクトル$\boldsymbol{r}_{2}=(x,y,z)=({\rm cos}\theta,{\rm sin}\theta,z)$とおく。このとき
\begin{align*}
    &\frac{\partial \boldsymbol{r}_{2}}{\partial z}=\begin{pmatrix}0\\0\\1\end{pmatrix}\\
    &\frac{\partial \boldsymbol{r}_{2}}{\partial \theta}=\begin{pmatrix}-{\rm sin}\theta\\{\rm cos}\theta\\0\end{pmatrix}
\end{align*}
となる。したがって、面積要素ベクトル$d\boldsymbol{S}$は
\begin{eqnarray*}
    d\boldsymbol{S}&=&\left(\frac{\partial \boldsymbol{r}_{2}}{\partial z}\times\frac{\partial \boldsymbol{r}_{2}}{\partial \theta}\right)dzd\theta\\
    &=&\begin{pmatrix}-{\rm cos}\theta\\-{\rm sin}\theta\\0\end{pmatrix}dzd\theta
\end{eqnarray*}
外向き正より、
\begin{eqnarray*}
    d\boldsymbol{S}=\begin{pmatrix}{\rm cos}\theta\\{\rm sin}\theta\\0\end{pmatrix}dzd\theta
\end{eqnarray*}
以上より、
\begin{eqnarray*}
    \iint_{x^2+y^2=1,-1\leq z\leq1} \boldsymbol{a}(\boldsymbol{r})\cdot d\boldsymbol{S}&=&\iint_{-1\leq z\leq 1,0\leq\theta\leq 2\pi} \begin{pmatrix}{\rm cos}\theta{\rm sin^2}\theta\\{\rm cos^2}\theta{\rm sin}\theta\\{\rm sin^2}\theta\end{pmatrix}\cdot\begin{pmatrix}{\rm cos}\theta\\{\rm sin}\theta\\0\end{pmatrix}dzd\theta\\
    &=&2\iint_{-1\leq z\leq 1,0\leq\theta\leq 2\pi} {\rm cos^2}\theta{\rm sin^2}\theta dzd\theta\\
    &=&\frac{1}{2}\iint_{-1\leq z\leq 1,0\leq\theta\leq 2\pi} {\rm sin^2}2\theta dzd\theta\\
    &=&\frac{1}{4}\int_{0}^{2\pi} (1-{\rm cos}4\theta)d\theta\int_{-1}^{1} dz\\
    &=&\pi
\end{eqnarray*}
(f-3)\\
$T=\left\{(r,\theta,z)|0\leq r\leq 1,0\leq \theta\leq\pi,-1\leq z\leq 1\right\}$である。
\begin{eqnarray*}
    \iiint_T (\mathrm{div}\boldsymbol{a})dv&=&\iiint_T \left(\frac{\partial (xy^2)}{\partial x}+\frac{\partial (x^2y)}{\partial y}+\frac{\partial y^2}{\partial z}\right)dxdydz\\
    &=&\iiint_T (x^2+y^2)dxdydz\\
    &=&\iiint_T r^3drd\theta dz\\
    &=&\int_{0}^{1} r^3dr\int_{0}^{2\pi} d\theta\int_{-1}^{1} dz\\
    &=&\pi
\end{eqnarray*}
となる。円柱の側面を$S_{1}$、上面を$S_{2}$、底面を$S_{3}$とおくと、
\begin{eqnarray*}
    \iint_{S_{2}} \boldsymbol{a}\cdot d\boldsymbol{S}=-\iint_{S_{3}} \boldsymbol{a}\cdot d\boldsymbol{S}=\frac{\pi}{4}
\end{eqnarray*}7
であることも踏まえると
\begin{eqnarray*}
    \iint_S \boldsymbol{a}(\boldsymbol{r})\cdot d\boldsymbol{S}&=&\iint_{S_{1}} \boldsymbol{a}\cdot d\boldsymbol{S}+\iint_{S_{2}} \boldsymbol{a}\cdot d\boldsymbol{S}+\iint_{S_{3}} \boldsymbol{a}\cdot d\boldsymbol{S}\\
    &=&\iiint_T (\mathrm{div}\boldsymbol{a})dv\\
    &=&\pi
\end{eqnarray*}
が成り立ち、ガウスの定理が成り立つことがわかる
\section*{問2}
\begin{screen}
    電磁気学における,マクスウェル方程式と電磁波の関係について調べ,ベクトル解析の知識を用いて記述せよ
\end{screen}
\end{document}